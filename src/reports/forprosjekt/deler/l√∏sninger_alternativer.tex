\chead{Løsninger/alternativer}
\chapter{Løsninger/alternativer}

\section {Foreslått løsningsmodell}
Under interaksjoner med produkteier og veiledere var det diskutert om vi skulle gå for en sentralisert eller en desentralisert løsning. Dette til tross for at Accenture sin oppgavetekst originalt hadde vært beskrevet en sentralisert løsningsmodell. På grunn av pandemien uttrykket de også at prosjektet ikke var helt komplett og derfor var de åpne om at gruppen kunne endre på selve oppgaven og til og med starte hele prosjektet på nytt igjen. Vår foreslåtte løsningsmodell er å utbedre det tidligere prosjektet, og bygge et sentralisert kommunikasjonssystem. Alternativene vi har diskutert er beskrevet nedenfor. 

\section{Alternative løsninger}
Oppgaven vi har blitt gitt er svært åpen, og det har gitt oss et stort rom for diskusjon rundt alternative løsninger. Oppdragsgiver har gjort det klart at gruppen har stor frihet til å velge den løsningen som virker best. Ettersom at prosjektet originalt var gjort av en tidligere bachelorgruppe, var et sentralt spørsmål om vi skulle fortsette på arbeidet deres eller å begynne fra bunnen av. Vi har også diskutert om hvorvidt de selvkjørende bilene, istedenfor å kommunisere gjennom et sentralisert kommunikasjonssystem, skal snakke med hverandre.

Å starte fra bunnen av ville innebært å bygge nye biler, lage ny kode for de selvkjørende bilene og også begynne arbeidet på det sentraliserte kommunikasjonssystemet som den forrige gruppen ikke kom i gang med. Fordelene ved å velge dette alternativet innebærer at gruppen vår har en bedre forståelse av koden, at gruppen sparer tid på å lese og forstå det tidligere prosjektets kode og gir potensielt mer mulighet for læring om kunstig intelligens og maskinlæring. En mulig konsekvens for dette alternativet at gruppen vår kommer frem til en relativt lik løsning som den tidligere gruppen. Dette mener vi ville vært et lite tilfredssstillende resultat for både bachelorgruppen, og for Accenture. Lik den tidligere bachelorgruppen, er det også mulig vi heller ikke får tid til å bygge det sentraliserte kommunikasjonssystemet dersom vi velger dette alternativet.

Det andre alternativet er å utbedre det tidligere prosjektet. Dette innebærer å sette seg grundig inn i det arbeidet som er blitt gjort, og videreutvikle det ved å utvikle kommunikasjon mellom bilene. En av de klareste fordelene er at vi sparer mye tid på å kunne gjenbruke koden og de fysiske modellene som allerede er laget. Vi tror også at dette alternativet lettere kan oppnå målet om å være et skalerbart system. En utfordring ved dette alternativet er at det potensielt kan være vanskelig å sette seg inn i kode som er skrevet av andre. 

I diskusjonen om hvorvidt vi skulle velge en desentralisert peer-to-peer-løsning eller et sentralisert kommunikasjonssystem, har vi vektlagt hva vi tror ville vært mest fordelaktig for Accenture. De ulike virkningene av disse valgene er diskutert grundigere under kapittelet om analyse av virkninger. 
