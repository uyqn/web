\chead{Presentasjon}
\chapter{Presentasjon}
Gruppen som skal utføre prosjekt hos Accenture som sitt bacheloroppgave er følgende dataingeniør studenter ifra OsloMet:

\begin{tabular}{lcc}
	Hansen, Andreas Torres & \href{mailto:s338851@oslomet.no}{s338851@oslomet.no} & Student \\
	Nguyen, Uy Quoc & \href{mailto:ss341864@oslomet.no}{s341864@oslomet.no} & Student\\
	Ottersland, Anders Hagen & \href{mailto:s341883@oslomet.no}{s341883@oslomet.no} & Student
\end{tabular}

Oppgaven som gruppen skal løse er en utvidelse av tidligere prosjekt utført av \cite{bachelor2020} hos Accenture. I år ønsker Accenture at gruppen skal implementere en IoT løsning slik at disse Raspberry Pi kjøretøyene kan kommunisere med et sentralisert system. Hovedfokuset til oppgaven vår er å bruke dette systemet for å demonstrere hvordan et slikt system kan forbedre trafikkflyten.

I tillegg har gruppen også fått utdelt veiledere og produkteier i de forskjellige institusjonene:

\begin{tabular}{lll}
	Zhang, Jianhua & \href{mailto:jianhuaz@oslomet.no}{jianhuaz@oslomet.no} & Veileder \\
	Fauske, Ivar Austin & \href{mailto:ivar.austin.fauske@accenture.com}{ivar.austin.fauske@accenture.com} & Veileder \\
	Johansen, Solfrid Hagen & \href{mailto:s.hagen.johansen@accenture.com}{s.hagen.johansen@accenture.com} & Veileder\\
	Vallestad, Benjamin & \href{mailto:benjamin.vallestad@accenture.com}{benjamin.vallestad@accenture.com} & Produkteier
\end{tabular}

