\chead{Arbeidsplan og fremdriftsplan}
\chapter{Arbeidsplan og fremdriftsplan}
\section{Arbeidsmetodikk}
\subsection{Gantt-diagram}
Per idag har vi laget et Gantt diagram med fornuftige tidsfrister og faser vi ønsker å forhold oss til. Desverre, så får vi ikke med et eksemplar av denne diagrammen på rapporten grunnet at funksjonaliteten er låst inntil vi betaler for den. Det kan tenkes at vi trenger å utforske hvilken prosjektstyring verktøy vi muligens kommer til å benytte oss av under resten av prosjektet.

Grovt sett ihenhold til Gantt diagrammet kan følgende grovt bli beskrevet:
\begin{enumerate}
    \item[\textbf{Fase 1:}] Utvidelse av Raspberry Pi kjøretøy
    \item[\textbf{Fase 2:}] Implementasjon av server
    \item[\textbf{Fase 3:}] Sikre kommunikasjon mellom klient og server.
    \item[\textbf{Fase 4:}] Integrering av maskin læring i serveren.
    \item[\textbf{Fase 5:}] Testing av løsning
\end{enumerate}
Igjennom hele prosjektet har vi også blitt enig om at utvikleren selv skriver enhetstestingen selv for å sikre kodekvalitet. Videre skal det utføres integrasjonstesting når fasene parvis er gjennomført.

\subsection{Kanban}
Medfølgende Gantt-diagrammet ovenfor har oppgavene også blitt lagt til en Kanban board. Disse oppgavene kan bli plukket opp og utviklet av en av medlemmene. Kategoriene vi har under Kanban er:
\begin{itemize}
    \item[\textbf{Open:}] Her ligger alle oppgaver som ikke har blitt startet på
    \item[\textbf{In progress:}] Oppgaver som har blitt startet av en utvikler ligger her.
    \item[\textbf{Testing:}] Oppgaver som er klar for testing legges her.
    \item[\textbf{Review:}] Oppgaver skal vurderes av alle andre i gruppen før den kan plasseres til neste Kategoriene.
    \item[\textbf{Closed:}] Her skal alle oppgaver som er ferdig legges   
\end{itemize}

\subsection{Scrum}
Gruppen ønsker å utnytte en agile metodikk på grunn av den åpne naturen av oppgaven. Vi tenker å dele prosjektperioden i sprinter men uten definerte roller siden gruppen er så liten. Dette vil gi oss muligheten til å iterere implementasjonen av løsningen der det kreves og for å få et overblikk av hvilke oppgaver som er ferdige og hvilke som trenger å bli reviewed, lukket eller startet på.

\subsection{Arbeidsdager og møte med veiledere}
Grunnet ekstra fag og deltidsarbeid har vi også bestemt faste arbeidsdager: Mandag, onsdag og fredag, mellom 9 og 16. Vi har også avtalt å alternere møter med veiledere henholdsvis annenhver uke med interneveileder og veiledere ifra Accenture.