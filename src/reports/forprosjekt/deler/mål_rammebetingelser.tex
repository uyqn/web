\chead{Mål og rammebetingelser}
\chapter{Mål og rammebetingelser}
\section{Mål}
Målet med dette prosjektet er todelt. I den første delen ønsker Accenture at prosjektets utstilling i konferanser og messer kan føre til økt attraksjon av fremtidelige arbeidssøkere. Dette var også nevnt i tidligere prosjekt av \cite{bachelor2020}. Videre var det uttrykket, etter at mange av deres konsulenter har oppdrag hos statens vegvesen, at prosjektet også kan bli presentert til deres klient for å generere interesse for å investere i et lignende prosjekt.

Mer konkret har vår produkteier uttalt følgende:
\begin{quotation}
    ``Målet er at man ender opp med en prototype som kan vises til interessenter for å demonstrere hvordan selvkjørende biler kan kombineres med et sentralisert system.
    Rammebetingelsene er at man lager noe som kan demonstreres, om det er digitalt eller fysisk blir opp til gruppen å bestemme. Det er ønskelig at prototypen kan demonstrere minst en situasjon som kan oppstå hvor utfallet blir ulikt ved bruk av selvkjørende + sentralisert vs kun selvkjørende. Systemet burde også kunne skaleres for å legge til eller fjerne enheter ved behov.''
\end{quotation}
\begin{flushright}
    - Benjamin Vallestad, Produkteier, 2022
\end{flushright}

\section{Rammebetingelser}
Rammebetingelsene er fortsatt ikke fastsatt av Accenture. De har uttrykket at oppgaven kan være åpent til vår tolkning og at gruppen kan selv bestemme hva de ønsker å gjøre med prosjektet. I henhold til det tidligere prosjektet var det visse punkter de ikke rakk å implementere. Gruppen ønsker derfor at disse kravene i dette prosjektet skal realiseres.

I henhold til den åpne naturen av oppgaven ønsker vi alikevel å begrense oss til å videreutvikle det den tidligere gruppen har laget. Dette har vi diskutert med både veileder og internt, for å forsikre oss at løsningen vår skaper verdi for Accenture og dermed forhindre at vi kommer fram til samme resultat som tidligere prosjekt.

\section{Teknologi}
Gruppen har diskutert og har kommet frem til et utkast av teknologier som vi tror vi ender opp med å bruke, med forbehold om at disse muligens kan endres iløpet av prosjektperioden dersom det skulle være hensiktsmessig.

\subsection{Git med Github}
Gruppen er sikker på at Git i samsvar med Github skal brukes til versjonhåndtering under prosjektet. Gruppen har blitt veldig komfortabel med dette verktøyet ved samarbeid av prosjekter igjennom studieløpet og ønsker å bruke dette til dette prosjektet også. Itillegg er Git også mye brukt i arbeidslivet.

\subsection{Python3 med Tensorflow}
I henhold til det tidligere prosjektet har \cite{bachelor2020} tatt i bruk Python3 med Tensorflow for å utvikle bilene. Siden gruppen skal fortsette på det prosjektet var dette en helt vanlig språk og bibliotek å velge for å videreutvikle bilene.

\subsection{C\# i backend}
Gruppen har diskutert hvilken språk vi skulle bruke i backend. Gruppen har tidligere utført prosjekt i med Python3 i backend og var enig om at språket var for treg i henhold til responstid og skalering. Vi ønsker et raskere språk for å redusere prosesseringstiden serverne måtte bruke for å sende tilbake en respons i potensielt kritiske situasjoner.

\subsection{Clickup for prosjektstyring}
Clickup er en webapplikasjon for prosjektstyring. Her er det mulig å organisere Gantt diagrammer og Kanban boards. Andreas har erfaring med dette verktøyet tidligere og gruppen ser at dette kan være en god verktøy for å styre prosjektet. En bonus er at dette er gratis i forhold til veldig mange andre prosjektstyring verktøy.

\subsection{\LaTeX for rapportskriving}
\LaTeX er ofte brukt for føring av vitenskapelige dokumenter. Verktøyet har mye automatisering av innholdsfortegnelse, bibliografi, tabeller og figurer på en ganske fornuftig måte. Tidligere rapporter var også skrevet i \LaTeX. Gruppen ser for seg at mye av formatering kunne være automatisert av verktøyet.

\subsection{Teams, Zoom og Discord}
I henhold til pandemien er det vanskelig med 100\% fysisk oppmøte. Mye av kommunikasjonen foregår digitalt og allerede nå har vi hatt forskjellige kommunikasjonskanaler ut ifra hvem vi skal kommunisere med.

Kommunikasjon med vår interneveileder foregår over Zoom. Dette er fordi dette har vært standard verktøy for skolen å benytte seg av under pandemien for å holde forelesninger osv.

Microsoft Teams har vært kommunikasjonskanalen som er benyttet av Accenture. Vi velger derfor også å bare fortsette å bruke Teams for kommunikasjon med veiledere og produkteier fra Accenture.

Intern kommunikasjon med teamet har vært vanlig for oss å gjennomføre igjennom Discord siden alle har tidligere erfaring med dette verktøyet igjennom gaming og tidligere prosjekter.