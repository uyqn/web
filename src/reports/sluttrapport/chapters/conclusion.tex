\chapter{Conclusion}
\section{Results}
In regards to the problem statement in \secref{sec:problem_statement} and the goals for this project discussed in \secref{sec:goals}, we have been able to successfully create a centralized system that can communicate with autonomous vehicles. A significant change in velocity can lead to a disruption of traffic flow, as discussed in \secref{sec:traffic_congestion}. Through multiple semi-physical demonstrations discussed in \secref{sec:demo}, it was apparent that the server was able to contribute to an improved traffic flow by preemptively reducing vehicles' to the necessary speed to prevent collisions. Even though this project did only account for scenarios involving intersections, it is also, from the exploration done throughout this project, clear that autonomous vehicles, together with a centralized system, can contribute to improving traffic flow in the future.

Furthermore, we received positive feedback after showing our supervisors and product-owner video snippets of the conducted demonstrations. The product owner also confirmed that our results adhered to the requirements set by Accenture.

%\subsection{Real world scenario}

If we compare the data from the semi physical demonstration to a real world cenario were instead of 80 cm/h the cars would drive in 80 km/h prior to an intersection. In the real world, one of the cars would have to stop before a traffic light while the other could drive past the intersection. This would lead to a velocity change from 80 km/h to 0. In our scenario the car would only have to slow down to 55 km/h.

However there are more factors to consider in a real world scenario such as: curved roads, human mistakes and animals jumping onto the road.
\subsection{Possible improvements}
All the features in the MoSCoW method from \secref{sec:moscow_method}, "could have" to "won't have this time" are features for a future project, either for Accenture to improve or for a future bachelor project. Although it fulfilled the product requirements given by Accenture, there is room for improvement. For example, with the accuracy of the demo, specifically by doing more calibration tests and making a more accurate formula than we were able to produce (see \hyperref[eq:vprelationship]{equation \eqref{eq:vprelationship}}). Furthermore, changing the wheels and installing an edge TPU to exploit the existing AI model will result in a more predictable trajectory of the vehicles.

Our demo only contains two cars, but our server supports scenarios with multiple cars. There are also possibilities to connect other devices to the server, for example, traffic lights. However, there are no specific functionalities for other IoT devices on the server. Scaling the IoT system for functionalities with other IoT devices is a task for future development. Adding roads and making a more complex road system would also be a task for future development.

As mentioned, we made a semi-physical simulation. However, a virtual simulation would potentially yield more data for research while also exploring other more complex road configurations.

Due to this project's time constraints, we, for instance, did not integrate machine learning with our server. Integrating machine learning on the server could expand its different capabilities and make the project more scalable and applicable to a broader range of scenarios. Such integration is deemed valuable for the further extension of this project.

Making a viable product in the real world is a long way ahead and will require a lot more testing and implementation on a bigger scale. With the rise in availability of self-driving vehicles and the 5G network, there could be a possibility of IoT systems managing traffic. We explored the posibility of implementing multiple servers, each  having responsibility for their own geographical area, or an entire road. This follows the edge computing architecture, which is popular in IoT systems and describes distributed network devices that communicate to one centralized server through ``edge'' devices \parencite[pp 149]{iot_platforms}. We did not implement this in our program, but it was one of our main ideas for how we would scale the system to implement more roads.

Finally, we hope that our testing and research can be of value towards realizing autonomous cars and traffic management on such a scale.

\section{Further discussions}
In addition to implementing the program, we also had discussions regarding the viability of the program. Here we will discuss our process and how such an IoT system would apply to the real world. We will use our data and some research to reflect the usability of an IoT system where self-driving cars can communicate through a centralized system.

\subsection{Self evaluation}
As mentioned earlier in \secref{sec:dev_method}, we took inspiration from two agile frameworks: Scrum and Kanban, but chose to lean more towards scrum in the end. The use of Scrum helped us reach our goals. However, we could have included our external supervisors more in the Scrum process in hindsight. We could have done this by including them in sprint retrospective meetings and discussing what our following sprint goals should have been together.

The IoT system can handle more vehicles, but the road model does not fully support functionalities for having more intersections than we currently have in our demonstration. Because of the time constraint, we also found it challenging to focus on scalability. Hence, we instead focused on getting a working demonstration.  

One challenge was that our group could not meet as much as we had wished because of work. However, all the requirements in the MosCoW analysis (\figref{fig:moscowmethod}) were reached except for the features in the "can have" section. All in all, the group was satisfied with the process.

\subsection{Educational Value}
Developing said program has been a challenging process, which our group has learned a lot from. We all feel that we have evolved into better developers, and that we now would be better suited for solving such a project in the future. 

\section{Further work}
How to implement self-driving in society in the best way is a question that will take a long time to answer. Through our work with this proof-of-concept we believe we have made an addition to this discussion. Due to time constraints, we have chosen not to implement some features that would make the product work in a more complex environment. These features could be explored if this project were to be further developed. 

\subsection{Scalability}
When the knowledge and research has come further regarding AI, there may be a possibility for such an IoT system to be scaled for the real world. There is a need for less traffic in the cities and our results from this project shows that an IoT-system where cars can communicate through a server will increase traffic flow. Scalability is therefore important which we had in mind while coding the server and the client.

Our demo only contains two cars but our server is built in a way where multiple vehicles can connect to it. There are also possibilities to connect other devices to the server, for example traffic lights. However there are no specific functionalities regarding traffic lights on the server as of now. Scaling the IoT system for functionalities with traffic lights is a task for future development. Adding roads and making a more complex road system would also be a task for future development. 

\subsection{Extendability to the real world applications}
\subsubsection*{Edge computing}
In the future when self-driving vehicles become more prominent, and the 5G network becomes more available there could be a possibility for IoT-systems handling traffic management. Our group therefore did some research regarding how the system will extend to the real world’s applications. 

The IoT systems usually follow the fog or edge computing architecture with distributed or even decentralized concepts \parencite[pp 149]{iot_platforms}. This is to prevent overloading on servers handling a lot of data. 

One solution to distribution of data is that each road has one server responsible for their respective road. If a road is long it will be split into geographical areas where one server has responsibility for their geographical area. Intersections will have a server handling information from both road’s respective servers, since information from both servers are needed to make decisions in intersections.

In the traffic there are a lot of unforeseen situations that can happen. Traditional coding will not be able to cover every outcome in a traffic situation, therefore there will be a need for AI on the servers in addition to the cars. 
\subsection{Risk management}