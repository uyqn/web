\chapter{Conclusion}
\section{Results}
In regards to the problem statement in \secref{sec:problem_statement} and the goals for this project discussed in \secref{sec:goals}, we have been able to successfully create a centralized system that can communicate with autonomous vehicles. A significant change in velocity can lead to a disruption of traffic flow, as discussed in \secref{sec:traffic_congestion}. Through multiple semi-physical demonstrations discussed in \secref{sec:demo}, it was apparent that the server was able to contribute to an improved traffic flow by preemptively reducing vehicles' to the necessary speed to prevent collisions. Even though this project did only account for scenarios involving intersections, it is also, from the exploration done throughout this project, clear that autonomous vehicles, together with a centralized system, can contribute to improving traffic flow in the future.

Furthermore, we received positive feedback after showing our supervisors and product-owner video snippets of the conducted demonstrations. The product owner also confirmed that our results adhered to the requirements set by Accenture.

\subsection{Real world scenario}

If we compare the data from the semi physical demonstration to a real world cenario were instead of 80 cm/h the cars would drive in 80 km/h prior to an intersection. In the real world, one of the cars would have to stop before a traffic light while the other could drive past the intersection. This would lead to a velocity change from 80 km/h to 0. In our scenario the car would only have to slow down to 55 km/h.

However there are more factors to consider in a real world scenario such as: curved roads, human mistakes and animals jumping onto the road.
\subsection{Possible improvements}
All the requirements in the MoSCoW method from "could have" to "won't have this time" are requirements for a future project, either for Accenture to improve or for a future bachelor project. Although it fulfilled the product requirements given by Accenture, there is room for improvement. In addition, there is room for improvement with the accuracy of the demo, specifically by doing more calibration tests and making a more accurate formula than we were able to produce. Furthermore, changing the wheels and installing an edge TPU to exploit the existing AI model will result in a more predictable trajectory of the vehicles.

Our demo only contains two cars, but our server supports scenarios with multiple cars. There are also possibilities to connect other devices to the server, for example, traffic lights. However, there are no specific functionalities regarding traffic lights on the server. Scaling the IoT system for functionalities with traffic lights is a task for future development. Adding roads and making a more complex road system would also be a task for future development.

As mentioned, we made a semi-physical demonstration. However, a virtual simulation will yield more data for research while also exploring other more complex road configurations.

Making a viable product in the real world is a long way ahead. That will require a lot more testing and implementation on a bigger scale. However, we hope that our testing and research can be of value towards that step.

\section{Further discussions}
In addition to implementing the program, we also had discussions regarding the viability of the program. Here we will discuss our process and how such an IoT system would apply to the real world. We will use our data and some research to reflect the usability of an IoT system where self-driving cars can communicate with each other.

\section{Self evaluation}

As mentioned earlier, we used inspiration from two agile frameworks: scrum and kanban, but chose to lean more towards scrum in the end. We felt the use of scrum helped us reach our goals. However, we could have included our external supervisors more in the scrum process in hindsight. We could have done this by including them in sprint retrospective meetings and discussing what our following sprint goals should have been together.

We satisfied most of the product goals and requirements, although some extra hours were needed towards the end. Because of the time constraint, we found it difficult to focus on scalability. The IoT system can handle more vehicles, but the road model does not fully support functionalities for having more roads than we currently have in our demonstration. We focused on getting a working demonstration rather than making it scaleable.  

One challenge in the process was that our group could not meet as much as we had wished because of work. If we had worked more throughout the process, the need for extra work, in the end, could have been prevented. In the MosCoW method (figure 3.1), we were able to finish all the requirements in "must have" and "should have" sections, but none of the "can have" sections. All in all, the group was satisfied with the process.

\subsection{Educational Value}
Developing said program has been a challenging process, which our group has learned a lot from. We all feel that we have evolved into better developers, and that we now would be better suited for solving such a project in the future. 

%\subsection{Real world application}

The best way to implement self-driving cars in society is a topic that will take a long time to explore fully. Through our work with this proof-of-concept, we believe we have made an addition to this discussion. Due to time constraints, we have chosen not to create a system that will be viable for all scenarios. Such a system might still seem out of reach. However, individual contributions and new additions to this discussion might take us one step closer to a complete system.


\subsection{Edge computing}
In the future when self-driving vehicles become more prominent, and the 5G network becomes more available there could be a possibility for IoT-systems handling traffic management. Our group therefore did some research regarding how the system will extend to the real world’s applications. 

The IoT systems usually follow the fog or edge computing architecture with distributed or even decentralized concepts \parencite[pp 149]{iot_platforms}. This is to prevent overloading on servers handling a lot of data. 

One solution to distribution of data is that each road has one server responsible for their respective road. If a road is long it will be split into geographical areas where one server has responsibility for their geographical area. Intersections will have a server handling information from both road’s respective servers, since information from both servers are needed to make decisions in intersections.

In the traffic there are a lot of unforeseen situations that can happen. Traditional coding will not be able to cover every outcome in a traffic situation, therefore there will be a need for AI on the servers in addition to the cars. 

System security

Security and privacy are important topics for any IoT-system. Because these systems gather and work with huge amounts of data, they are naturally prone to being attacked. And as the systems grow and become more interconnected, with many devices around the world, the imposed risk of such an attack increases drastically [kilde på dette].

Therefore a lot of security measures needs to be implemented in the real world application.


%\input{chapters/sections/subsections/risk_management}
\subsection{System security}
Security and privacy are important topics for any IoT system. Because these systems gather and work with vast amounts of data, they are naturally prone to be attacked. Moreover, as the systems grow and become more interconnected, with many devices worldwide, the imposed risk of such an attack increases drastically \parencite{iot_risk}. Anonymization of personal data, securing connections, and an intruder detection system are all security measures that require attention. The cars must be able to drive both with or without the system, which is an essential feature of system security. If there is a need for the server to shut down, the cars would need to be able to drive independently of the system.
% \subsection{Distribution to the real world application}