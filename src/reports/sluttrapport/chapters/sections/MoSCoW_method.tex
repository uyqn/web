\section{Prioritization method}\label{sec:moscow_method}
The MoSCoW method is a prioritization technique used in project management. The word MoSCoW is an acronym where:

\begin{itemize}
	\item "Mo" stands for must-have and represents our project's most prioritized requirements. These are necessary for the success of our project.
	\item "S" stands for should have. Our project should include these features, but they are not mandatory.
	\item "Co" stands for could have. We want to include these features in our project, but they are not prioritized.
	\item "W" stands for will not have this time. The features in the "W" section might be for a later group if someone wants to build on our project further.
\end{itemize}

Since we were unsure of how many features we could finish within the time frame of the project period, we thought a prioritization method was a good fit. \figref{fig:moscowmethod} shows our visualization of the Moscow method.

\begin{figure}[h!]
	\centering
	\includegraphics[width=1\linewidth]{figures/MosCoW_method}
	\caption[MosCoW method]{Visualuzation MoSCoW method. Everything under "Mo" are requirements we must have in our project. Under the "S" sections are features we should have. Under "Co" are features we could have but are not necessary. Under "W" are features we will not have this time}
	\label{fig:moscowmethod}
\end{figure}


