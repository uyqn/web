\section{Internet of Things (IoT)}
The Internet Of Things refers to physical objects that communicate with the use of sensors, cameras, software or other technologies that connect and exchange data. This communication takes place over the Internet or other communication forms. The number of connected IoT-devices in the world is increasing, and it is becoming a big part of society \parencite{iot_analytics}. The field of IoT has also been evolving in recent years due to other technologies becoming more accessible, such as machine learning.

You need not look further than to the smart-home consumer market to find applications in your life, of devices communicating to solve problems. It could also be applied in climate surveillance systems, energy or transportation. The benefits that an internet of connected devices could add billions in value to industries across the world, and to the global economy. In this thesis we will explore the possibilities of using IoT in transportation, more specifically in personal automobiles. The convergence of these fields is more commonly known as IoV, Internet of Vehicles, and it is a central theme of our thesis. An IoV system is a distributed system for wireless communication and information exchange between vehicles through agreed upon communication protocols \parencite{chinese_iov}.  The system could potentially integrate functionality for dynamic information exchange, vehicle control and  smart traffic management. In our thesis we will explore these possibilities on a small scale, with the hopes of making a solution that can be scaled up at a later point. 

Challenges to be aware of with IoT-systems, and consequently also IoV-systems include, among others, ethical questions about decision-making, physical and digital safety for humans and infrastructure, storage of data and power usage. With the evolution, and increased availability of artificial intelligence (AI), IoT has also evolved to implement algorithms powered by AI. We will discuss the implementation of AI and machine learning further down in the thesis, but it is possible to view some of the problems with IoV-systems as an extension of typical problems with AI systems. In our work with the thesis we have kept these challenges in mind when formulating proposed solutions, and it has affected how we have worked.

