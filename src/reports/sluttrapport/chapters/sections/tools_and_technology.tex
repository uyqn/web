\section{Tools and technologies}
The circumstances surrounding Covid 19 meant that we could not meet our supervisors in person at the start. Luckily, the group could still meet physically a few times a week. We had to use a wide range of tools for communication. Most restrictions were removed later in the project, but we kept our meetings with our supervisors digital throughout the project. 

\begin{itemize}
	\item Email - Formal communication with supervisors and product owner
	\item Teams- Meetings, and the platform of choice for communicating with the external supervisors on an informal level. 
	\item Zoom- Meetings
	\item Messenger - communication internally in the group. We used it to send messages to each other when we were not physically together, and to send pictures of code. 
\end{itemize}

The project required us to collaborate while working on different personal computers, which can lead to overlapping. Therefore, tools that helped us work on the project together were essential. Here are the tools that helped us:

\begin{itemize}
	\item Git and Github - Version control of choice
	\item Google docs - Used to write our journal and other documents that need to get updated regularly, and to share documents. 
\end{itemize}

We also needed text editors that supported the programming languages that we used. The client was built in python, while we built the server in C#. 

\begin{itemize}
	\item Pycharm- Text IDE for coding in python
	\item Visual studio code- IDE for coding in C\#
	\item Thonny - Text editor for coding in python on Raspberry pi
\end{itemize}

Project planning and documentation were also an important part of the project. The tools we used for the project planning were:

\begin{itemize}
	\item Github project- Kanban board and creating backlog tasks
	\item Excel- Used for visualizing our worklan by using tables and a gantt diagram
\end{itemize}