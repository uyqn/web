\section{Tools and technologies}
The circumstances surrounding Covid 19 ment that we were not able to meet our supervisors in person at the start, however many of the restrictions cleared up at the later part of the project. Luckily the group was still able to meet physically a few times a week. We still had to use a wide range of tools for communication.
\begin{itemize}
	\item Email - Formal communication with supervisors and product owner
	\item Teams- Meetings, and the platform of choice for communicating with the external supervisors on an informal level. 
	\item Zoom- Meetings
	\item Messenger - communication internally in the group. We used it to send messages to each other when we were not physically together, and to send pictures of code. 
\end{itemize}

The project required us to collaborate while working on different personal computers which can lead to overlapping. Therefore, tools that helped us to work on the project together were important. Here is the tools that helped us:

\begin{itemize}
	\item Git and Github - Version control of choice
	\item Google docs - Used to write our journal and other documents that need to get updated regularly, and to share documents. 
\end{itemize}

We also needed text editors that supported our programming languages that we used. The client was built in python while the server was built in C\#. 

\begin{itemize}
	\item Pycharm- Text IDE for coding in python
	\item Visual studio code- IDE for coding in C\#
	\item Thonny - Text editor for coding in python on Raspberry pi
\end{itemize}

Project planning and documentation was also an important part of the project. Here is a few tools we used for the project planning:

\begin{itemize}
	\item Github project- Kanban board and creating backlog tasks
	\item Excel- Used for visualizing our worklan by using tables and a gantt diagram
\end{itemize}