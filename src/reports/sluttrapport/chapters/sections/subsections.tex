\section{Work methodology}
As previously mentioned, we chose to apply an agile work methodology based on the fact that our project seemed to be prone to many changes. We included rituals like daily stand-ups, sprint planning meetings and retrospective meetings after sprints. Doing this helped our team understand what difficulties the other members were facing, and gave us time to discuss our obstacles daily. 

There were also some elements of agile development that we did not incorporate, either because of lack of experience or lack of time. As explained in the work of  Skyttermoen, T. and Vaagasar, A.L.; agile development teams regularly have meetings with the product owner, who give feedback on the strengths and weaknesses of the deliveries \textcolor{red}{(Skyttermoen & Vaagaasar, 2017, pp.120)}. Our group only had a few meetings with the product owner during the project period, to update him on how far along we were. We did, however,  have biweekly meetings with our external supervisors at Accenture who provided us feedback on the work we had done. Because we had the freedom to choose the type of solution we wanted, as long as it met the requirements that were set to us, we didn’t need the product owner to be in our biweekly meetings. 