\section{Self evaluation}

As mentioned earlier, we used inspiration from two agile frameworks: scrum and kanban, but chose to lean more towards scrum in the end. We felt the use of scrum helped us reach our goals. However, we could have included our external supervisors more in the scrum process in hindsight. We could have done this by including them in sprint retrospective meetings and discussing what our following sprint goals should have been together.

We satisfied most of the product goals and requirements, although some extra hours were needed towards the end. Because of the time constraint, we found it difficult to focus on scalability. The IoT system can handle more vehicles, but the road model does not fully support functionalities for having more roads than we currently have in our demonstration. We focused on getting a working demonstration rather than making it scaleable.  

One challenge in the process was that our group could not meet as much as we had wished because of work. If we had worked more throughout the process, the need for extra work, in the end, could have been prevented. In the MosCoW method (figure 3.1), we were able to finish all the requirements in "must have" and "should have" sections, but none of the "can have" sections. All in all, the group was satisfied with the process.

\subsection{Educational Value}
Developing said program has been a challenging process, which our group has learned a lot from. We all feel that we have evolved into better developers, and that we now would be better suited for solving such a project in the future. 
