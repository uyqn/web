\section{Phase 2: REST API}\label{phase2}
The aim of the project was to connect Raspberry Pi devices to a system. Thus, the group decided that the best way to establish this connection was to implement a RESTful API on the server. In addition, the question on how the data should be stored was also arises during this phase.

\emph{Representational state transfer} (REST) \emph{application programming interface} (API) provides a way for client and server to establish communication through \emph{hypertext transfer protocol} (HTTP). Using a REST API clients can send requests to a server to perform standard CRUD (create, read, update and delete) operations on a database \parencite{rest_api}.

Due to the time sensitivity of the application we are trying to build, it was therefore necessary to choose an appropriate type of database for our system. In this case, we chose to incorporate the time series database. InfluxDB is a time series database created by InfluxData. It provides a SQL-like syntax for querying resources that is quick and scalable, and most importantly free. Moreover, InfluxDB client library, using the influxDB v2 API, provides both ease of install and use for a multitude of langauages \parencite{influxdb}.

During this phase the group implemented a standard REST API server with C\# with the idea that the Raspberry Pi vehicles should exchange information on its velocity, acceleration and position to the server. Client should first POST itself to the server. The server will then add the vehicle to the influxDB to keep track of the vehicle's information. Then, the vehicle should be able to perform a GET request to the server to retrieve its information.

The client at this phase would be able to send a PATCH request to the server in order to update its information. In addition, the server will add a new entry into the database whenever it receives this request. The idea was that the client and server would be able to continuously communicate to each other such that the server could determine the behaviour of all connected clients.

However, a RESTful API server were not able to perform all its required task that the group wanted the server to do. Firstly, the server were only able to communicate with one client at the time, i.e. the client that sends a request. What the group wanted at this stage was that based on a request the server should also be able to send its own request to all the other clients. In order to acheive this, the server had to send an unprompted response to other clients that is not requesting a resource from the server, which was not possible with our current architecture.

A new solution had to be in place in order to achieve our goal. Other solutions were proposed during this stage.

\subsubsection*{Long polling}
Polling is the idea that the server pushes resources to the client. There are mainly two types of polling; short- and long polling. 

In short polling, a client requests a resource from the server and the server responds with nothing if the resrouce is not available. The client will then send a new request in a short amount of time and the cycles repeats until the client receives the resource it has requested.

Long polling is similar to short polling, however, the server does not send anything back before the resource is available. That is, the client sends a request to the server and the server is holding this request until it has a response available to the client. In our case, we wanted every client to perform a GET request to the server on a seperate thread and instruct the server to hold onto this request until it had further instructions to the requesting client.

However, implementing a method on the server to block the response introduced more complication to the project. Also, with the asynchronous nature of the controllers implemented on the server it would also mean that the server will consume a lot of the processor which also means that the performance of the server will be heavily deteriorated.

\subsubsection*{Implement REST API on the client}
Another solution was to implement the client itself as a REST API server on its own. However, in order to achieve this, each client needed to also send the server its host and port information to the server. Also, the server has to be implemented as a client in order to connect to the vehicles.

\subsubsection{Webhooks}
Webhooks, according to \cite{webhooks}, is a user-defined callback over HTTP. In our case, implementing webhooks to post notifications on clients based on events sent to the server. This was a good contender to solve our issue. However, implementing webhooks includes extensive research into a system the group had never heard of, in addition to scarce information on how to create such a system. The group decided that the time constraint of this project did not justify the time it would take to implement such a system.
