
\section{Phase 2 - REST API and why it did not work with our project}\label{phase2}
The project aimed to connect Raspberry Pi devices to a server.  Our initial approach was to implement a RESTful API on a server, as the connection layer between them. We also considered data security a critical aspect of an upscaled version of such a system. The group discussed how to store the data about the vehicle's positions during this phase, keeping in mind that we wanted our proof-of-concept to be scalable to real life.

\emph{Representational state transfer} (REST) \emph{application programming interface} (API) provides a way for clients and servers to establish communication through \emph{hypertext transfer protocol} (HTTP), which is a protocol for transferring data between network devices. Using a REST API and the standard HTTP methods; POST, GET, PATCH, DELETE, clients can send requests to a server to perform standard CRUD (create, read, update and delete) operations on a database respectively \parencite{rest_api}. We wanted to keep track of all connected cars, on a database on the server. Therefore, REST API seemed like a good fit for our current solution.

Due to the time constraint of the program development, quickly choosing an appropriate type of database for our system was desirable, as migrating databases later could be a big timesink. We chose to incorporate a time-series database, which automatically includes a timestamp for each database entry. InfluxDB is a time-series database created by InfluxData and provides SQL-like syntax that is quick to query resources. Moreover, it provides both ease of installation and supports a multitude of languages \parencite{influxdb}.

During this phase, the group implemented a standard REST API server with C\# with the idea that the Raspberry Pi vehicles should exchange information on their velocities, accelerations, and positions to the server. When initially connecting to the server, the client should first provide the information of its properties, through a POST call. The server will then add the vehicle to the InfluxDB to keep track of the vehicle's information. The vehicle should be able to perform a GET request to the server to retrieve its information.

At this stage, the client will send PATCH requests to the server to update its information. In addition, the server will add a new entry to the database whenever it receives this request. The idea was that the client and server would be able to continuously communicate with each other, creating a live feed of the clients' behavior.

However, it became apparent that a RESTFul API could not achieve the results that we wanted. Firstly, the server was only able to communicate with one client at a time, i.e., the client that sends a request. The group wanted the server to respond to other clients without a request. In our case, whenever a vehicle sends a PATCH request, the server should be able to inform other vehicles of this event.

A new solution had to be in place to achieve our goal, so we consulted our supervisors for help. They proposed some solutions for us to research, including long-polling, Webhooks, and Webscokets discussed in \secref{longpolling}, \secref{webhooks} and \secref{websockets}, respectively.

\subsection{Alternative solution 1 - Short-polling and long-polling}\label{longpolling}
Polling refers to the server pushing resources to the client. There are mainly two types of polling; short- and long-polling. 

When short-polling, a client requests a resource from the server, and the server responds with an empty response if the resource is not available. The client will then send a new request after a short amount of time, and the cycle repeats until the client receives the resource it has requested.

Long-polling is similar to short-polling, but the server does not send anything back until the resource is available. In other words, the client sends a request to the server, and the server holds this request until it has a response available to the client. In our case, we wanted every client to perform a GET request to the server. Then the server holds onto this request until it has further instructions for the requesting client.

\subsection{Alternative solution 2 - Webhooks}\label{webhooks}
Webhooks, according to \cite{webhooks}, is a user-defined callback over HTTP. In our case, implementing webhooks to post notifications on clients based on events sent to the server. This was a good contender to solve our issue. However, implementing webhooks includes extensive research into a system the group had never heard of, in addition to scarce information on how to create such a system. The group decided that the time constraint of this project did not justify the time it would take to implement such a system. We found little information on how to utilize webhooks in our project.                                                                                                                                                                                                                       

\subsection{Alternative solution 3 - WebSockets}\label{websockets}
Websockets is a protocol that provides a bidirectional communication between clients and server by establishing a single TCP connection in both direction \parencite{rfc_websockets}.  
