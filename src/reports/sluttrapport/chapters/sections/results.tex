\section{Results}

We observed thoughout multible semi physical demoes, that even if the cars startet at the same distance from an intersection and had the same initial velocity, they would be able to pass the intersection without having to stop. This is because prior to the intersection the server would tell the car that would enter the intersection last to slow down. An example of a successful demo kan be seen in figure 3.6. According to what the server was logging, the car that slowed down prior the the intersection had changed its velocity from 80cm/s so 55c cm/s. The car had then returned to 80cm/s after the other car had passed the intersection.

If we compare that data to a real world cenario were instead of 80 cm/h the cars would drive in 80 km/h prior to an intersection. In the real world, one of the cars would have to stop before a traffic light while the other could drive past the intersection. This would lead to a velocity change from 80 km/h to 0. In our scenario the car would only have to slow down to 55 km/h.

Change of velocity can lead to a distruption of traffic flow. We can  therefore conclude that in the specific scenario showed in the semi physical demonstration the server improved traffic flow.   

\section{Evaluation of process}
In our process documentation we have described the way the project progressed from beginning to end. In this section we will discuss positive and negative takeaways, with the goal of aiding further work on the project. This will include our approach to work methodology, communication and team building. 

\subsection{Work methodology}
As previously mentioned, we chose to apply an agile work methodology based on the fact that our project seemed to be prone to many changes. We included rituals like daily stand-ups, sprint planning meetings and retrospective meetings after sprints. Doing this helped our team understand what difficulties the other members were facing, and gave us time to discuss our obstacles daily. 

There were also some elements of agile development that we did not incorporate, either because of lack of experience or lack of time. As explained in the work of  Skyttermoen, T. and Vaagasar, A.L.; agile development teams regularly have meetings with the product owner, who give feedback on the strengths and weaknesses of the deliveries \textcolor{red}{(Skyttermoen \& Vaagaasar, 2017, pp.120)}. Our group only had a few meetings with the product owner during the project period, to update him on how far along we were. We did, however,  have biweekly meetings with our external supervisors at Accenture who provided us feedback on the work we had done. Because we had the freedom to choose the type of solution we wanted, as long as it met the requirements that were set to us, we didn’t need the product owner to be in our biweekly meetings. 
\subsection{Possible improvements}
All the features in the MoSCoW method from \secref{sec:moscow_method}, "could have" to "won't have this time" are features for a future project, either for Accenture to improve or for a future bachelor project. Although it fulfilled the product requirements given by Accenture, there is room for improvement. For example, with the accuracy of the demo, specifically by doing more calibration tests and making a more accurate formula than we were able to produce (see \hyperref[eq:vprelationship]{equation \eqref{eq:vprelationship}}). Furthermore, changing the wheels and installing an edge TPU to exploit the existing AI model will result in a more predictable trajectory of the vehicles.

Our demo only contains two cars, but our server supports scenarios with multiple cars. There are also possibilities to connect other devices to the server, for example, traffic lights. However, there are no specific functionalities for other IoT devices on the server. Scaling the IoT system for functionalities with other IoT devices is a task for future development. Adding roads and making a more complex road system would also be a task for future development.

As mentioned, we made a semi-physical simulation. However, a virtual simulation would potentially yield more data for research while also exploring other more complex road configurations.

Due to this project's time constraints, we, for instance, did not integrate machine learning with our server. Integrating machine learning on the server could expand its different capabilities and make the project more scalable and applicable to a broader range of scenarios. Such integration is deemed valuable for the further extension of this project.

Making a viable product in the real world is a long way ahead and will require a lot more testing and implementation on a bigger scale. With the rise in availability of self-driving vehicles and the 5G network, there could be a possibility of IoT systems managing traffic. We explored the posibility of implementing multiple servers, each  having responsibility for their own geographical area, or an entire road. This follows the edge computing architecture, which is popular in IoT systems and describes distributed network devices that communicate to one centralized server through ``edge'' devices \parencite[pp 149]{iot_platforms}. We did not implement this in our program, but it was one of our main ideas for how we would scale the system to implement more roads.

Finally, we hope that our testing and research can be of value towards realizing autonomous cars and traffic management on such a scale.
