\section{Implementation of simulation}
To show that the solution was reaching the requirements we needed to create a demonstration. A demonstration was also an important part of testing the functionalities of the IoT-system. There were two options which were viable in this case. The first one was to create a virtual simulation using unity or another graphic-program. The other option was to build a physical demonstration with two or more cars. We chose to implement a physical demonstration with two cars. This was because we already had one car finished by the previous group. In addition it was more beneficial for Accenture with a physical demo since they can show it at exhibits.
\begin{itemize}
	\item That the server can turn off and the cars would still drive on their own.
	\item Cars communicating through the server and acing based on the server’s decisions
	\item Scalability to the real world
\end{itemize}
First we built the server with only one single lane road in mind. With this solution the potential of showing off the functionalities of the IoT-system was low. We could potentially have shown that if a car in front slows down, the car behind also slows down. This would show that the system could prevent some traffic, but only in a specific scenario. However we wanted to show that the IoT system could work in more than just one single-lane road. We therefore chose to try to emulate an intersection in the physical demo.

For a physical demo to work we needed to build another car. Luckily the previous group had documented their work and we could follow their process from their final report. We also had the parts provided to us. However we were unable to get the full functionality of the second car since the TPU accelerator used by the previous group was unavailable. The TPU accelerator provided the car with the processing power needed for the AI. This meant that the car was built without the camera. But this had a minor effect on our demo since the other car was fully functional without the system. We solved this problem by letting the server do the decision making for the car with less functionality.

We also needed to emulate a road system which contained the intersection we were going to demonstrate in the demo. Since the AI had not trained to recognize intersections yet we had to code a road system into our server that contained roads, lanes and intersections. We then had to build a physical intersection for our cars with tape that correlates with the emulated intersection. 

\subsection{Calibration of the cars}
When the server, vehicles and client were implemented, and the second vehicle built, we did some testing to figure out how the car’s behaved when given directions by the server. In this test the vehicles were given a specific velocity and driving distance by the server. When the vehicles arrived at their destination the server would tell them to stop. The car’s drove in a straight line.

The vehicles were able to send information, and respond correctly to the servers commands. We also observed that the vehicles drove a different length for each velocity given even though the length was the same. This is because the velocity given to the vehicles is the amount of power going into the car’s motors, not the actual velocity of the car’s. We wanted the demo to be accurate so our group did some further testing where we wrote down the results. 

\begin{table}
	\begin{center}
		\begin{tabular}{rrrr}
			\hline
			Power (?) & Length (cm) & Time (s) & Velocity (cm/s) \\
			\hline
			40 & 467 & 8.98 & 52.00 \\
			50 & 425 & 7.28 & 58.38 \\
			60 & 400 & 6.06 & 66.01 \\
			70 & 357 & 5.18 & 68.92 \\
			80 & 325 & 4.49 & 72.32 \\
			90 & 314 & 4.03 & 77.92 \\
			100 & 286 & 3.62 & 79.01 \\
			\hline
		\end{tabular}
		\caption{Test text}
	\end{center}
\end{table}

The data Power was the velocity given by the server. Velocity was the actual velocity in our testing, which is length divided by time. As you can see the velocity was not the same as the power. We then made a graph to visualize the two values. The $y$-axis was the velocity while the $x$-axis is the power.

\begin{figure}[h!]
	\caption{Graph of velocity as a function of power}
\end{figure}

We observed that the correlation between power and velocity seemed linear. This means we could make a specific formula that describes the correlation between the two values. We used linear regression to figure out this formula:

\begin{figure}[h!]
	\caption{Graph of velocity as a function of power with linear regression}
\end{figure}

The formula we ended up with was as follows: $P = 0.4516v + 36.189$, where $P$ is power and $v$ is velocity, with a mean square error of $R^2=0.9653$. When we coded the formula into the vehicles we did another set of testing. We observed that the vehicles drove more or less the same distance for each power given. If we wanted an even more accurate formula we could have tuned the formula with the test results from our new test. Although the results were not hundred percent accurate, we concluded it was accurate enough for our demonstration. 

To test the solution we have worked on, we made a physical demonstration with two cars that meet at an intersection, as part of the product documentation. We want to test that a combination of a centralized communication system and artificial intelligence can improve traffic flow. What we wanted to observe was if the velocity of the vehicles were not drasticly changed and therefore not distrupting the traffic flow.

