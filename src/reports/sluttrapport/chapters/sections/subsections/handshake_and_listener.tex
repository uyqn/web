\subsection{Handshake and listener}\label{handshake}
After initializing the vehicle class as a client with
\begin{python}
async def main():
	...
	client = Vehicle()
	...
\end{python}
then the client's \verb|listen| method can be called:
\begin{python}
async def main():
	...
	listener = asyncio.create_task(client.listen())
	...
\end{python}
The listener method is responsible for handling responses and requests from the server. Hence, it must run concurrently as the vehicle continuously sends data to the server.

When the listener is called, the client performs the following code:
\begin{python}
class Client:
...
	async def listen(self):
	async with websockets.connect(self.websocket_uri) as websocket:
		self.__websocket = websocket
		await self.__handshake()
		await self.__listen()
...
\end{python}
The method first opens a WebSocket connection using the stored URI and stores this as a private variable. Then, a handshake with the server is performed:
\begin{python}
class Client:
...
	async def __handshake(self, protocol: str = "json", version: int = 1):
		data = self.signalr_encode_message({"protocol": protocol, "version": version})
		await self.__websocket.send(data)
		response = self.signalr_decode_message(await self.__websocket.recv())
		if "error" in response:
			print(response)
		else:
			await self.send_non_blocking("AddClient", self.properties)
...
\end{python}

The code above describes the handshake process between the client and the server. First, the client informs the server of the protocols it will use throughout its lifetime. Then, it stores the client, in this case, the vehicle, to the server using the defined properties.

Further elaboration, the client invokes the method
\begin{csharp}
public partial class VehiclesHub : Hub
{
	...
	public async Task AddClient(JsonDocument jsonDocument) {...}
	...
}
\end{csharp}
on the server. This method first creates a vehicle with all the provided information sent by the client
\begin{csharp}
public partial class VehiclesHub : Hub
{
	...
	public async Task AddClient(JsonDocument jsonDocument)
	{
		...
		var vehicle = Vehicle.Create(jsonDocument);
		...
	}
	...
}
\end{csharp}
using the static method defined by the Vehicle model. In addition, it assigns the travel plan to the vehicle by using values defined by config.json from the client:
\begin{json}
{
	...
	"vehicle": {
		...
		"travel_plan": {
			"start": {
				"road": 0,
				"lane_reversed": false
			},
			"end": {
				"road": 0,
				"lane_reversed": false
			}
		}
	}
}
\end{json}

Furthermore, the vehicle's current lane is also assigned to track which lane the vehicle is driving on. Then, \verb|public class VehiclesHubDatabase| adds the \verb|Vehicle|, together with its connection id \verb|Context.ConnectionId| for easy retrieval.

Moreover, for the sake of the demo, the server is also instructed to wait for a second vehicle to connect before allowing the vehicles to drive
\begin{csharp}
public partial class VehiclesHub : Hub
{
	...
	public async Task AddClient(JsonDocument jsonDocument)
	{
		...
		vehicle.Velocity = 0;
		_database.Update(vehicle);
		await Clients.Client(Context.ConnectionId).SendAsync("adjust_velocity", vehicle);
		await WaitForVehicles(vehicle, _database.SpeedLimit, 2);
	}
	...
}
\end{csharp}
by first setting the vehicle's velocity to zero, updating the new velocity in the database, and adjusting the velocity of the client to zero. Lastly, it calls the \verb|WaitForVehicles| method, which will adjust every client's velocity to the defined \verb|SpeedLimit| in \verb|VehiclesHubDatabase|.
