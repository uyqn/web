\subsection{Possible improvements}

There are many factors to consider in a real-world scenario, such as curved roads, human mistakes, and animals jumping onto the road. Therefore the data extracted from the demonstrations can not perfectly correlate with a real-world scenario. 

All the requirements in the MoSCoW method from "could have" to "won't have this time" are requirements for a future project, either for Accenture to improve or for a future bachelor project. Although fulfilled the product requirements given by Accenture, there is room for improvement. In addition, some improvements can be made with the accuracy of the demo, specifically by doing more calibration tests and making a more accurate formula than we were able to produce. A more accurate demonstration can also be made by changing some of the wheels so that the vehicles can drive more straight.

Our demo only contains two cars, but our server is built in a way where multiple vehicles can connect to it. There are also possibilities to connect other devices to the server, for example, traffic lights. However, there are no specific functionalities regarding traffic lights on the server. Scaling the IoT system for functionalities with traffic lights is a task for future development. Adding roads and making a more complex road system would also be a task for future development. 

As mentioned, we made a semi-physical demonstration. To extract more data, making a virtual simulation would be sufficient. In a virtual simulation, multiple scenarios could be tested with more vehicles.  

Making a viable product in the real world is a long way ahead. That will require a lot more testing and implementation on a bigger scale. However, we hope that our testing and research can be of value towards that step.