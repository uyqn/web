\subsection{Possible improvements}
All the requirements in the MoSCoW method from "could have" to "won't have this time" are requirements for a future project, either for Accenture to improve or for a future bachelor project. Although it fulfilled the product requirements given by Accenture, there is room for improvement. In addition, there is room for improvement with the accuracy of the demo, specifically by doing more calibration tests and making a more accurate formula than we were able to produce. Furthermore, changing the wheels and installing an edge TPU to exploit the existing AI model will result in a more predictable trajectory of the vehicles.

Our demo only contains two cars, but our server supports scenarios with multiple cars. There are also possibilities to connect other devices to the server, for example, traffic lights. However, there are no specific functionalities regarding traffic lights on the server. Scaling the IoT system for functionalities with traffic lights is a task for future development. Adding roads and making a more complex road system would also be a task for future development.

As mentioned, we made a semi-physical demonstration. However, a virtual simulation will yield more data for research while also exploring other more complex road configurations.

Making a viable product in the real world is a long way ahead. That will require a lot more testing and implementation on a bigger scale. However, we hope that our testing and research can be of value towards that step.