\subsection{Possible improvements}
All the features in the MoSCoW method from \secref{sec:moscow_method}, "could have" to "won't have this time" are features for a future project, either for Accenture to improve or for a future bachelor project. Although it fulfilled the product requirements given by Accenture, there is room for improvement. For example, with the accuracy of the demo, specifically by doing more calibration tests and making a more accurate formula than we were able to produce (see \hyperref[eq:vprelationship]{equation \eqref{eq:vprelationship}}). Furthermore, changing the wheels and installing an edge TPU to exploit the existing AI model will result in a more predictable trajectory of the vehicles.

Our demo only contains two cars, but our server supports scenarios with multiple cars. There are also possibilities to connect other devices to the server, for example, traffic lights. However, there are no specific functionalities for other IoT devices on the server. Scaling the IoT system for functionalities with other IoT devices is a task for future development. Adding roads and making a more complex road system would also be a task for future development.

As mentioned, we made a semi-physical simulation. However, a virtual simulation would potentially yield more data for research while also exploring other more complex road configurations.

Due to this project's time constraints, we, for instance, did not integrate machine learning with our server. Integrating machine learning on the server could expand its different capabilities and make the project more scalable and applicable to a broader range of scenarios. Such integration is deemed valuable for the further extension of this project.

Making a viable product in the real world is a long way ahead and will require a lot more testing and implementation on a bigger scale. However, we hope that our testing and research can be of value towards that step.