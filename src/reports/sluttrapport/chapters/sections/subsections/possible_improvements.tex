\subsection{Possible improvements}

Although we were able to forfill the product requirements there are room for improvement. All the requirements in the MoSCoW method from "could have" to "won't have this time" are requirements for a future project, either for accenture to improve or for a future bachelor project. In addition there are also improvements that can be done with the accuracy of the demo, specificly by doing more calibration tests and make a more accurate formula than we were able to produce. A more accurate demonstration can also be made by changing some of the wheels so that the vehicles can drive more straight.

Our demo only contains two cars but our server is built in a way where multiple vehicles can connect to it. There are also possibilities to connect other devices to the server, for example traffic lights. However there are no specific functionalities regarding traffic lights on the server as of now. Scaling the IoT system for functionalities with traffic lights is a task for future development. Adding roads and making a more complex road system would also be a task for future development. 

As mentioned we made a semi-physical demonstration. For the sake of getting more data a viritual simulation could be made. Here multible scenarios could be tested with more vehicles.  

Making a product that i viable in the real world is a long way ahead. That will require a lot more testing and implementation on a bigger scale. However, we hope that our testing and research can be of value towards that step.