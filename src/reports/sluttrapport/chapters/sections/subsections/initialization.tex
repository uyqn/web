\subsection{Initialization}\label{initialization}
\subsubsection{Client.py}
The client package is an essential module in the Raspberry Pi vehicles for the success of this project. The client class is mainly responsible for connecting, handling, and sending data to the server. The client class does not aim to be used alone but rather as a superclass for other IoT devices. Hence, it should be inherited and handle everything that pertains to client-server communication in the background.

The client's init method does several things: It reads from the config.json file to store the defined host and port it will use to establish a connection to the server.
\begin{python}
class Client:
	def __init__(self, properties=None, **kwargs):
		...
		with open("client/config.json") as f:
			config = json.load(f).get('client')
			if config is not None:
				self.__uri = f"://{config.get('host')}:{config.get('port')}/{self.__class__.__name__.lower()}sHub"
				self.__delay = config.get('delay')
		...
\end{python}

Then, it starts a negotiation process with the server, where it receives a connection id that the client will use during its connection lifetime to the hub.

\begin{python}
class Client:
	def __init__(self, properties=None, **kwargs):
		...
		urllib3.disable_warnings()
		response = requests.post(f"http{self.__uri}/negotiate?negotiateVersion=0", verify=False)
		self.connection_id = response.json().get("connectionId")
		self.websocket_uri = f"ws{self.__uri}?id={self.connection_id}"
		...
\end{python}

The client also gives itself a random id stored as one of its properties. The client's id is also stored on the server and is mainly used to retrieve and update the client's information.

Furthermore, \verb|Client.__init__| also stores a dictionary of events.
\begin{python}
class Client:
	def __init__(self, properties=None, **kwargs):
		...
		self.subscribed_events = {
			"disconnect": self.disconnect,
			"force_patch": self.force_patch,
			"continuously_patch": self.continuously_force_patch
		}
		...
\end{python}

Values in this dictionary are references to functions in this class and is used to invoke certain behaviours by the server. For instance, \newline\verb|await Clients.Client(Context.ConnectionId).SendAsync("disconnect");| from the server will call \verb|def disconnect(self)| in the client. 

\subsubsection{Vehicle.py}
The vehicle class is supposed to contain all the data and methods of the vehicle. Furthermore, \verb|class Vehicle(Client)| inherits the client class which enables \verb|Vehicle| to perform all the necessary operations to establish connection upon initiation. An important remark is that \verb|Client| performs the negotiation to the server using endpoint \newline \verb|{self.__class__.__name__.lower()}sHub/negotiate?negotiateVersion=0| meaning that through inheritence and initialization of \verb|Vehicle|, the subclass negotiates with the endpoint  \verb|vehiclesHub/negotiate?negotiateVersion=0|, which is mapped in \verb|Program.cs| with
\begin{csharp}
...
app.MapHub<VehiclesHub>("/vehiclesHub");
...
\end{csharp}

Furthermore, \verb|Vehicle| also reads from \verb|config.json| to define it's initial properties with the snippet shown below:
\begin{python}
class Vehicle(Client):
	def __init__(self, properties=None, **kwargs):
		...
		if properties is None and len(kwargs) == 0:
			with open("client/config.json") as f:
				config = json.load(f).get('vehicle')
				if config is not None:
					self.properties.update(config)
		...
\end{python}

\verb|Vehicle| also utilizes the property builder of \verb|Client|.

\begin{python}
class Vehicle(Client):
	def __init__(self, properties=None, **kwargs):
		...
		self.property_builder(
			required={'length', 'height', 'width', 'mass'},
			optional={'velocity': 0, 'position': 0, 'travel_plan': None},
		)
		...
\end{python}

In short, the property builder is used to define the required properties of the vehicle class. The meaning is to restrict somewhat what data the vehicle class should contain. If one should directly initialize Vehicle without using config.json, one must assign values to length, height, width, and mass.

Lastly, \verb|Vehicle| adds subscribed events that the server can invoke:
\begin{python}
class Vehicle(Client):
	def __init__(self, properties=None, **kwargs):
		...
		self.subscribed_events.update({
			"adjust_velocity": self.adjust_velocity
		})
\end{python}

Likewise, as in \verb|Client|, should other events be required for a vehicle, one can add it to the dictionary as proposed above.