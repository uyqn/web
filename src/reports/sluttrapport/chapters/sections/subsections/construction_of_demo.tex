\subsection{Construction of semi physical demonstration}\label{sec:demo}
We want to test that a centralized communication system and artificial intelligence can improve traffic flow. In order to test the solution we have worked on, we made a physical demonstration where two cars approach an intersection simultaneously. We wanted to observe if the velocity of the vehicles were not drastically changed and, therefore, decreased shockwave described in \secref{sec:traffic_congestion}

We found a space at Accenture that was big enough to build the track. The power input on the Raspberry Pi vehicles is limited to between 40 and 100. Since the server will adjust the vehicle's speed to avoid a collision, the intersection was required to be at least 130cm away from the starting point to prevent the server from adjusting speed outside the equivalent power limit. Furthermore, the server prevents cars from driving before at least two vehicles have established a connection. Hence, both vehicles will start their journey simultaneously. We also placed the vehicles at the same distance from the intersection on their respective roads. Given these initial conditions, both vehicles are supposed to collide without the server's intervention. We also made the server log the velocity sent to the cars to track how much the cars' velocities changed.

Making the demo one hundred percent accurate was not possible in our circumstances due to the limitations of the Raspberry Pi. Furthermore, the vehicles were not able to receive the messages simultaneously. Consequently, this meant that they could start with a minor time difference. Another factor was that the trajectory of the vehicles was not always straight. However, we were able to get a consistent demo with enough margins. \figref{fig:crashdemo} shows an example of a collision during a test demonstration.

\begin{figure}[h!]
	\centering
	\includegraphics[width=1\linewidth]{figures/demo_crash}
	\caption{Here, we can observe the two cars colliding in the intersection. One of the cars started about 5 centimeters further behind the other car and started a few milliseconds later, resulting in a collision. Meanwhile, the server assumed they started simultaneously at the same distance from the intersection.}
	\label{fig:crashdemo}
\end{figure}

Furthermore, the server did not account for the lengths of the vehicles during its calculations. After we introduced the length of the vehicles and buffer zone, we were able to get a consistent semi-physical demonstration.

When both vehicles had connected to the server, the cars would drive with an initial speed of 80 cm/s, the upper limit of the Raspberry Pi. Not long after they started to drive, the server recognized the cars approaching the intersection. The server calculates using the car's velocity, position, and length. Then the server calculates which car has to slow down and how much the car needs to slow down to avoid a collision, in this case, to 55 cm/s. After the other car has supposedly passed the intersection, the car that slowed down gets told by the server to speed its velocity back to 80 cm/s. \figref{fig:successdemo} is a snapshot of a successful test demo.

\begin{figure}[h!]
	\centering
	\includegraphics[width=1\linewidth]{figures/succsess_demo}
	\caption[Successful demo]{Here is a snipped from a successful demo. The car to the left has just passed the intersection, marked as a square with white tape. The car furthest up is, therefore, about to adjust back to its original velocity. Here we can observe that the server prevents a collision.}
	\label{fig:successdemo}
\end{figure}