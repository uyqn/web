\subsection{Implementation of Client}
Raspberry Pi is a small single-board desktop computer that is commonly used for IoT-projects. There is an enormous ecosystem of compatible devices that allow these computers to interact with the world in various ways. The organization states that they can be used for everything “from music machines and parent detectors to weather stations and tweeting bird houses with infra-red cameras” \parencite{raspberrypi}.  For our project we have used a Raspberry Pi Model 4, which the previous group used in their project. This is the newest and fastest model, which makes our test results as accurate as possible.  

The perception layer is responsible for perceiving the world, and creating data for the network layer to collect and deliver \parencite[pp. 8-9]{iot_platforms}. Devices that contribute to this are, for example, Global Positioning Systems(GPS), cameras and sensors. The raspberry-pi models we worked with utilizes both a camera and a range detection sensor that gets processed by the image recognition algorithm running on the machine. These vehicles play the role of client in our system, as they connect to the server. 

The requirements for our client was that it needed to be able to take in commands from the server and respond correctly to those commands. In addition, they had to be able to act by themselves if they were not connected to the server. For that to happen the vehicles had to send their size, position and velocity to the server when they connected initially. The server also needs to send their velocity and position to the server at a frequent rate so that the server can keep track of that information.