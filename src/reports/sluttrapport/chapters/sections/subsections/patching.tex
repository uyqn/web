\subsection{Patch}
After initialization and handshake elaborated in \hyperref[initialization]{\ref{initialization} Initialization} and \hyperref[handshake]{\ref{handshake} Handshake and listener} respectively, the Raspberry Pi vehicles starts to drive into an intersection simultaneously. Throughout the journey the cars are continuously patching to the server, by calling the client's \verb|async def send\_patch| method.

\begin{python}
class Client:
	...
	async def send_patch(self, **kwargs) -> None:
		if self.properties_has_changed(**kwargs) or self.__continuously_patch:
			await self.send_invocation("patch", self.properties)
		else:
			await asyncio.sleep(self.__delay)
\end{python}

As seen above \pythoninline{send\_patch} calls the \pythoninline{async def send_invocation} method, which communicates the vehicle's current information by invoking \newline
\verb|public async Task Patch| on \verb|VehiclesHub|.

The patch method on \verb|VehiclesHub| is responsible for handling the behaviour, specifically adjusting the velocity of individual vehicles:
\begin{csharp}
public partial class VehiclesHub : Hub
{
	...
	public async Task Patch(JsonDocument jsonDocument)
	{
		var vehicle = Vehicle.Create(jsonDocument);
		_database.Update(vehicle);
		vehicle = _database.Fetch(vehicle);
		...
	}
}
\end{csharp}
The snippet above shows that the method first creates a new vehicle using the information provided by the \verb|Client|. However, since this new vehicle does not contain all the information, such as the travel plan, the method first update the existing vehicle in the database in order to refresh the vehicle with the available information. It then fetch the same vehicle that was stored in the handshake, mentioned in \hyperref[handshake]{\ref{handshake} Handshake and listener}. Assuming that the vehicle has been successfully retrieved it will then handle this vehicle accordingly:
\begin{csharp}
public partial class VehiclesHub : Hub
{
	...
	public async Task Patch(JsonDocument jsonDocument)
	{
		...
		await HandleIntersection(vehicle);
		await HandleInsideIntersection(vehicle);
		await HandleEndOfRoute(vehicle);
		...
	}
}
\end{csharp}
Shortly summarized \verb|HandleIntersection| is responsible to adjust the velocity of every vehicle approaching the intersection to avoid collisions. Furthermore, \verb|HandleInsideIntersection| increases the speed to \verb|VehiclesHubDatabase| defined \verb|SpeedLimit|. Lastly, \verb|HandleEndOfRoute| ensure that any vehicles that has completed their journey, defined during the handshake, terminates their connection with the server.