\subsection{Prioritization method}
The Moscow method is a prioritization technique used in project management to prioritize requirements. The word is split into four sections: The "Mo" stands for must-have and represents our project's most prioritized requirements. These are necessary for the success of our project. "S" stands for should have. Our project should include these requirements, but they are not mandatory. The "Co" stands for could have. The requirements in this section are requirements that we want to include in our project but chose not to prioritize. Lastly, the "W" stands for will not have this time. The requirements in the "W" section might be for a later group if someone wants to build on our project further.

Since we were unsure of how many use cases we could finish within the time frame of the project period, we thought a prioritization method was a good fit. Figure 3.2 shows our visualization of the Moscow method.

\begin{figure}[h!]
	\centering
	\includegraphics[width=1\linewidth]{figures/MosCoW_method}
	\caption[MosCoW method]{Visualuzation MoSCoW method. Everything under "Mo" are requirements we must have in our project, under the "S" sections are requirements we should have, and under "Co" are requirements we should have, however not necessary. Under "W" are requirements we will not have this time}
	\label{fig:moscowmethod}
\end{figure}


