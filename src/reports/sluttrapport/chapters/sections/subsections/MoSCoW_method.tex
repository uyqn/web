\subsection{Prioritization method}
The MoSCoW method is a prioritization technique used in project management. The word MoSCoW is an acronym:

\begin{itemize}
	\item "Mo" stands for must-have and represents our project's most prioritized requirements. These are necessary for the success of our project.
	\item "S" stands for should have. Our project should include these requirements, but they are not mandatory.
	\item "Co" stands for could have. We want want to include these requirements in our project but they are not prioritized.
	\item "W" stands for will not have this time. The requirements in the "W" section might be for a later group if someone wants to build on our project further.
\end{itemize}

Since we were unsure of how many use cases we could finish within the time frame of the project period, we thought a prioritization method was a good fit. Figure 3.2 shows our visualization of the Moscow method.

\begin{figure}[h!]
	\centering
	\includegraphics[width=1\linewidth]{figures/MosCoW_method}
	\caption[MosCoW method]{Visualuzation MoSCoW method. Everything under "Mo" are requirements we must have in our project, under the "S" sections are requirements we should have, and under "Co" are requirements we should have, however not necessary. Under "W" are requirements we will not have this time}
	\label{fig:moscowmethod}
\end{figure}


