\subsection{Prioritization method}

MoSCoW method is a prioritization technique used in project managment to prioritize requirements. The word is split into four sections: The "Mo" stands for must have and, is the number one priority requirements for our project. Theese are necesarry for the success of our project. "S" stands for should have. Theese requirements should be incuded in our project, however not mandetory. The "Co" stands for could have. The requirements in this section are requirements that are extras we want to include in our project, however not prioritized. At last the "W" stands for will not have this time. The requirements in the "W" section is maybe for a later group if someone wants to further build on our project.

 Since we were unsure how many use cases we finish within the time contraint of the project period, we thought a prioritazion method was a good fit.  Under is the visualization of the MoSCoW method.

\begin{figure}[h!]
	\centering
	\includegraphics[width=1\linewidth]{figures/MosCoW_method}
	\caption[MosCoW method]{Visualuzation MoSCoW method. Everything under "Mo" are requirements we must have in our project, under the "S" sections are requirements we should have, and under "Co" are requirements we should have, however not necessary. Under "W" are requirements we will not have this time}
	\label{fig:moscowmethod}
\end{figure}


