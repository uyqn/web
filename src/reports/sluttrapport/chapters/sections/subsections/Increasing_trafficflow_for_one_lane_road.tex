\subsection{Preventing traffic congestions for a one-lane road}

Traffic congestion, also known as traffic jams, is when a long line of vehicles moves slowly or has stopped moving altogether. Traffic jams can create frustration and disrupt nearby local environments with sound and gas emissions \parencite{traffic_congestion_pollution}. Many factors can cause traffic congestion, such as:
poorly designed roads, not wide enough roads, traffic light patterns, and accidents \parencite{traffic_congestion}.

<<<<<<< HEAD
With this in mind, we started by focusing on a simple scenario: when a car drastically reduces its speed or completely stops on a single-lane road 

This scenario will lead to the vehicles behind needing to slow down drastically as well. This phenomenon is called traffic jam shockwave \parencite{traffic_shockwave}. To prevent this, we propose a solution where cars reduce their velocity before they reach the destination of where the shockwave started. For this to happen, a server could keep track of the cars' positions and send information to the vehicles behind, when required. 
=======
With this in mind, we started by focusing on a simple scenario: when a car drastically reduces its speed or completely stops on a single-lane road.

This scenario will lead to the vehicles behind needing to slow down drastically \parencite{traffic_shockwave}. Because the human reaction is not perfect, this can lead to multiple cars having to emergency brake, disrupting the traffic flow. To prevent this, the vehicles behind have to decrease their velocity before they reach the destination of where the event happened. For this to happen, a server needs to keep track of the positions of the cars and send information to the cars behind when needed. 
>>>>>>> 69104b09590f1fac90cbed4f86b362032fec5193

We came up with an idea on how the server and cars should interact. First, the car would connect to the server and provide information about its current speed, weight, width, and length. The server would use this information to keep track of all the cars' positions on the road. The cars would send information to the server if their velocity changed. This message would trigger an event on the server where it would command all the cars behind to slow down accordingly. \figref{fig:diagramfirst} shows a flow chart of a potential simulation of this solution:

\begin{figure}[h!]
	\centering
	\includegraphics[width=0.9\linewidth]{figures/flow_diagram_first}
	\caption[Flow diagram server]{This figure shows the flow diagram of our first proposed solution. }
	\label{fig:diagramfirst}
\end{figure}