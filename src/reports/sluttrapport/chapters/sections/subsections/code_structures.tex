\subsection{Client.py}
The client package is the main module in the Raspberry Pi vehicles. The class that is mainly responsible to connect, handle and sending data to the server is the client class in client.py. Client class is not meant to be used alone but rather as a super class for other IoT devices. Hence, it was developed with the intention to be inherited and handle everything that pertains to client-server communication in the background.

Client's init method does several things: It reads from the config.json file to store the defined host and port it is going to connect to.
\begin{python}
with open("client/config.json") as f:
	config = json.load(f).get('client')
	if config is not None:
	self.__uri = f"://{config.get('host')}:{config.get('port')}/{self.__class__.__name__.lower()}sHub"
	self.__delay = config.get('delay')
\end{python}

Then, it starts a negotiation proccess with the server where it receives a connection id that the client will use during its connection lifetime to the hub.

\begin{python}
urllib3.disable_warnings()
response = requests.post(f"http{self.__uri}/negotiate?negotiateVersion=0", verify=False)
self.connection_id = response.json().get("connectionId")
self.websocket_uri = f"ws{self.__uri}?id={self.connection_id}"
\end{python}

The client also gives itself a random ID that is stored as one of its properties. The client's id is also stored on the server and is mostly used to retrieve and update the client's information on the server.

Furthermore, \pythoninline{Client.__init__} also stores a dictionary of events.
\begin{python}
self.subscribed_events = {
	"disconnect": self.disconnect,
	"force_patch": self.force_patch,
	"continuously_patch": self.continuously_force_patch
}
\end{python}

The values of this dictionary is a reference to a function in this class and is used to invoke certain behaviours by the server. For instance, \csharpinline{await Clients.Client(Context.ConnectionId).SendAsync("disconnect");} from the server will call \pythoninline{def disconnect(self):} in the client. 