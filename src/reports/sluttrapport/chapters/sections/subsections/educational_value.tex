\subsection{Educational Value}
Developing said program has been a challenging process, which our group has learned a lot from. We all feel that we have evolved into better developers, and that we now would be better suited for solving such a project in the future. 

The project description we were given was very open to interpretation, which gave us a lot of room for exploration. We chose to go with an IoT-solution using the car Accenture already owned, and also building a new one. This choice has given us significant insight into the making of IoT-systems and ways to set up communication between them. We believe this resulted in a much more interesting demonstration for Accenture to showcase. Further, how to combine such a system this with artificial intelligence has been a very interesting learning possibility. Due to the time constraints of such a project we, for instance, did not get to develop our own AI model, that could have helped the server make decisions based on optimal rulesets. This is an improvement that can be considered for further work. 

While we still were in the first development phase, we decided to adopt the code of the previous group, instead of developing our own AI model. We also believe that this was the right choice, as we believe it would have been a big time sink to start from scratch. Although it was a challenge to understand the program the previous group had written our assumption was that this way we would get to focus more on our demonstration. The importance of good documentation has also been made clear to us, which has benefited our own documentation. 

Our group had little to no experience working with an agile work methodology. In retrospective we believe we benefitted from this choice. Our daily stand ups allowed us to have a clear vision of the group's collective challenges. Although we did not implement all aspects of scrum-development or kanban-development we have gained insight into how a small development team can structure and plan out its work flow in an agile manner, that enables frequent changes in g