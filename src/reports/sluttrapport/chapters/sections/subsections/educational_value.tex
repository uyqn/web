\subsection{Educational Value}
Developing this solution has been challenging, and our group has learned a lot. We all feel that we have evolved into better developers and that we now would be better suited for solving similar projects in the future. 

The project description we started with was very open to interpretation, which gave us much room for exploration. We chose to go with an IoT solution using the car the prevoius group made. Ultimately, we also decided to build a new one. This choice has given us significant insight into making IoT systems and ways to set up communications between them. We believe this resulted in a much more exciting demonstration for Accenture to showcase.

While we still were in the first development phase \secref{sec:phase1}, we decided to adopt the code of the previous group instead of developing our own AI model. We believed that this was the right choice, as we assumed it would have consumed too much time to start from scratch. Although it was a challenge to understand the program the previous group had written, we assumed that this way, we would get to focus more on our demonstration. The importance of documentation has also been emphasized, which helped our group to write our report. 

Our group had little to no experience working with an agile work methodology. In retrospect, we believe we benefitted from this choice. Our daily stand-ups allowed us to have a clear vision of the group's collective challenges. Although we only implemented some aspects of Scrum and Kanban methodology, we have gained insight into how a small development team can structure and plan out its workflow in an agile manner.