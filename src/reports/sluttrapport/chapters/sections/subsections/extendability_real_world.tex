\subsection{Edge computing and AI}
When self-driving vehicles become more prominent and the 5G network becomes more available, there could be a possibility for IoT systems to handle traffic management. Therefore, our group researched how the system will extend to the real world's applications. 

The IoT systems usually follow the fog or edge computing architecture with distributed or even decentralized concepts to prevent overloading on servers handling vast loads of data \parencite[pp 149]{iot_platforms}. 

One solution to data distribution is that each road has one server responsible for its respective road. If a road is long, it will split into geographical areas where one server has responsibility for their geographical area. Intersections will have a server handling information from both roads' respective servers since the information from both servers is needed to make decisions.

In traffic, there are a lot of unforeseen situations that can happen. Traditional coding will not be able to cover every outcome in a traffic situation. Therefore there will be a need for AI on the servers and the cars. The AI will probably need to train in a safe test environment just like the autonomous vehicles before they are ready for existing road systems.

In our IoT program, the server gives commands to the cars, which overrides the cars' AI. In contrast, ideally, an interaction between the AI on the server and the cars is required for real-world scenarios.

