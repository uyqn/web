
\subsection{Edge computing}
In the future when self-driving vehicles become more prominent, and the 5G network becomes more available there could be a possibility for IoT-systems handling traffic management. Our group therefore did some research regarding how the system will extend to the real world’s applications. 

The IoT systems usually follow the fog or edge computing architecture with distributed or even decentralized concepts \parencite[pp 149]{iot_platforms}. This is to prevent overloading on servers handling a lot of data. 

One solution to distribution of data is that each road has one server responsible for their respective road. If a road is long it will be split into geographical areas where one server has responsibility for their geographical area. Intersections will have a server handling information from both road’s respective servers, since information from both servers are needed to make decisions in intersections.

In the traffic there are a lot of unforeseen situations that can happen. Traditional coding will not be able to cover every outcome in a traffic situation, therefore there will be a need for AI on the servers in addition to the cars. 

System security

Security and privacy are important topics for any IoT-system. Because these systems gather and work with huge amounts of data, they are naturally prone to being attacked. And as the systems grow and become more interconnected, with many devices around the world, the imposed risk of such an attack increases drastically [kilde på dette].

Therefore a lot of security measures needs to be implemented in the real world application.

