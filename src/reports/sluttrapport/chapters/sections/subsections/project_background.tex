\subsection{Project background}
In 2020, Accenture supervised a bachelor's student group to develop a self-driving car. The group developed a model-sized vehicle that uses artificial intelligence to analyze data from sensors and cameras and makes decisions based on the data. Accenture wants to build further on this project by exploring the addition of a centralized communication system for the vehicles.

The project's product owner states various reasons that make it relevant for Accenture. According to Accenture, Norway is one of the countries most ready to start utilizing self-driving cars, although not all factors that need to be in place are ready. Self-driving cars alone will also likely not be able to solve all the problems related to today's traffic problems. Accenture wants to explore if combining self-driving cars with an autonomous management system could provide greater value. 


==================

Accenture has in many years offered final year students at OsloMet and Høyskolen i Kristiania interesting projects for their bachelor thesis. In 2020, a group of students from Høyskolen i Kristiania were developing model-sized self driving vehicles using Raspberry Pi in conjunction with machine learning as their project. This year our group was offered to extend this project further; to explore plausible improvement with the addition of a centrilized communication system.

Norway is one of the countries that are ready to utilize self-driving cars, according to Accenture. However, self-driving cars alone is likely not enough to solve all of today's traffic challenges. Hence, the purpose of this project is an attempt to solve the issue by introducing a management system for autonomous vehicles and evaluate the value such system can provide.