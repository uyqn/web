\documentclass[11pt, a4paper, norsk]{report}
\usepackage[utf8]{inputenc}
\usepackage[T1]{fontenc,url}
\usepackage{babel,textcomp}
\usepackage{amsmath, amssymb}
\usepackage{polynom}
\usepackage[parfill]{parskip}
\usepackage{titlesec}
\usepackage{fancyhdr}
\usepackage{natbib}
\usepackage{hyperref}
\hypersetup{
	colorlinks   = true, %Colours links instead of ugly boxes
	urlcolor     = blue, %Colour for external hyperlinks
	linkcolor    = blue, %Colour of internal links
	citecolor   = red %Colour of citations
}

\begin{document}
\pagenumbering{gobble}
\section*{
\centering Bachelorprosjekt våren 2022 ved OsloMet\\
Prosjektskisse
}
\textbf{Sted og dato:} Oslo, 4. november 2021

\textbf{Foreløpig tittel:} IoT - oppgave

\textbf{Gruppenummer:} 23

\begin{tabular}{lll}
    \textbf{Medlemmer:}\\
    Andreas Torres Hansen & Uy Quoc Nguyen & Anders Hagen Ottersland \\
	\href{tel:+4746649739}{+47 466 49 739} & \href{tel:+4747684020}{+47 476 84 020} & \href{tel:+4741251290}{+47 412 51 290}\\
	\href{mailto:s338851@oslomet.no}{s338851@oslomet.no} & \href{mailto:s341864@oslomet.no}{s341864@oslomet.no} & \href{mailto:s341883@oslomet.no}{s341883@oslomet.no}
\end{tabular}

\begin{tabular}{l}
    \textbf{Skolens interne veileder:}\\
    Professor\\
    Jianhua Zhang\\
    \href{tel:+4767236691}{+47 672 36 691}\\
    \href{mailto:jianhua.zhang@oslomet.no}{jianhua.zhang@oslomet.no}
\end{tabular}

\begin{tabular}{l}
    \textbf{Oppdragsgiver:}\\
    Accenture\\
    Rolfsbuktveien 2, 1364, Fornebu,\\
    Norge, 1326
\end{tabular}


\begin{tabular}{ll}
    \textbf{Kontaktpersoner hos oppdragsgiver:}\\
    \textbf{Manager:}& \textbf{Consultant:}\\
    Marius Torsrud & Daniel Meinecke\\
    \href{tel:+4791854004}{+47 918 54 004} & \href{tel:+4797672525}{+47 976 72 525}\\
    \href{mailto:marius.torsrud@accenture.com}{marius.torsrud@accenture.com} & \href{mailto:daniel.meinecke@accenture.com}{daniel.meinecke@accenture.com}
\end{tabular}

\subsection*{Beskrivelse:}
Accenture er en internasjonal konsern som spesialiserer seg innenfor IT og konsultasjon tjenester. De har lang erfaring med å tilby studenter, per nå fra Høyskolen Kristiania og OsloMet, å skrive bacheloroppgave innenfor relevante IT studier hos dem. I år har de lagt ut en rekke spennende oppgaver for et bachelorprosjekt. Accenture har dermed tilbudt gruppen vår å løse deres IoT-oppgave.

Ifjor hadde en gruppe bachelorstudenter bygget to kjøretøy som benytter Raspberry Pi til å styre bilene på egenhånd. I år ønsker Accenture å videreutvikle dette konseptet slik at kjøretøyene skal kunne kjøre på et sentralisert system. Prosjektet krever derfor at man setter seg inn i tidligere skrevet kode og utvikle det sentraliserte systemet fra bunnen av. Dette systemet skal kommunisere med IoT-bilene som er utstyrt med sensorer og kamera for å planlegge rutene til biler i et større nettverk.
\end{document}