\documentclass[12pt, a4paper, norsk]{article}
\usepackage[utf8]{inputenc}
\usepackage[T1]{fontenc,url}
\usepackage{babel,textcomp}
\usepackage{amsmath, amssymb}
\usepackage{polynom}
\usepackage[parfill]{parskip}


% ------------------------------------------------------------
% Side formatering
\usepackage{fancyhdr}
\lhead{DATA3900}
\chead{Statusrapport}
\rhead{Gruppe 23}
\usepackage{lastpage}
\cfoot{Side \thepage\hspace{.5pt} of \pageref{LastPage}}
% ------------------------------------------------------------
% ------------------------------------------------------------
% Egne kommandoer for matematiske uttrykk:

%% ------------------------------------------ %% 

% ------------------------------------------------------------
\title
{
	DATA3900 - Bacheloroppgave \\
	Statusrapport
}
\author{Gruppe 23:\\
	Nguyen, Uy Quoc (s341864)\\
	Ottersland, Anders Hagen (s341883)\\
	Hansen, Andreas Torres (s338851)\\
	\\Antall sider: \pageref{LastPage}}
\date{Sist oppdatert:\\ \today}

\begin{document}
	\maketitle
	\pagenumbering{gobble}
	\clearpage
	\pagestyle{fancy}
	\pagenumbering{arabic}
	
	\section{Gruppe medlemmer}
	Denne bacheloroppgaven skal løses av følgende gruppe medlemmer:
	\begin{itemize}
		\item Nguyen, Uy Quoc (s341864) (Dataingeniør)
		\item Ottersland, Anders Hagen (s341883) (Dataingeniør)
		\item Hansen, Andreas Torres (s338851) (Dataingeniør)\\
	\end{itemize}
	Gruppemedlemmene har jobbet tidligere sammen på flere vellykkede prosjekter. I betrakning til god sammarbeid og kjemi mellom medlemmene har disse medlemmene valgt å danne gruppe til bacheloroppgaven også. Denne gruppen har meldt seg inn i gruppe 23 i faget.
	
	\section{Foretrukket prosjekt type og bedrift}
	Gruppen er veldig fleksible på teknologi og type prosjekt for denne bacheloroppgaven. Men, mangel på øyne for design foretrekker gruppen å unngå prosjekter som involverer for mye frontend. Ellers er gruppen veldig raske på å lære seg ny teknologi og foretrekker å delta i prosjekter som krever implementasjon av backend løsninger. Dette gjelder også hvor gruppen vil jobbe. I henhold til bedrift/virksomhet spiller det ikke så stor rolle så lenge det ikke kreves en altfor stor involvering av frontend løsninger.
	
	\section{Nåværende status}
	Gruppen har prøvd å nå ut til flere bedrifter igjennom jobbsøknadsannonser via epost og bekjente. Uheldigvis har få bedrifter hatt muligheten til å gi noe tilbakemelding på de forsøkte kontaktene. De bedriftene som gruppen har vellykket etablert kontakt med har avvist gruppen. Gruppen kommer til å fortsette å kontakte flere bedrifter så lenge det er gunstig i henhold til tidsfristen men ønsker gjerne intervensjon av utdanningsinstituttet for hjelp.
	
	
\end{document}